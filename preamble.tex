%%余白設定
\usepackage[top=25truemm,bottom=25truemm,left=25truemm,right=25truemm]{geometry}

%ヘッダー・フッター
\usepackage{fancyhdr}
\pagestyle{fancy}
\lhead{\leftmark}
\rhead{\bf\thepage}
\cfoot{}
\renewcommand{\chaptermark}[1]{\markboth{第\ \normalfont\thechapter\ 章\quad~#1}{}}

%%見出し深さ変更
\setcounter{tocdepth}{2}
%%章,節表記変更
\renewcommand{\prechaptername}{第}
\renewcommand{\postchaptername}{章}
\renewcommand{\thesection}{\arabic{chapter}.\arabic{section}}
\renewcommand{\thesubsection}{\arabic{chapter}.\arabic{section}.\arabic{subsection}}
% \renewcommand{\thesubsubsection}{(\alph{subsubsection})}

%%数式
\usepackage{amsthm,amsmath, amssymb}

%%図
\usepackage[dvipdfmx]{graphicx}
\graphicspath{ {./figs/} }
\usepackage[font={small}]{caption}
\usepackage[dvipdfmx]{color}

%%サブキャプション入りの図貼り込み
\usepackage[format=hang,font=small,labelfont=bf,labelsep=space]{caption}
\usepackage[subrefformat=parens]{subcaption}
\captionsetup{compatibility=false}

%%引用部を上付きカギ括弧に
\usepackage{cite}
\makeatletter 
\def\@cite#1{\textsuperscript{#1)}}
\renewcommand*{\@biblabel}[1]{#1)\hfill}%
\makeatother 

%%参照文献URL表記
\usepackage{url}
\urlstyle{same}

%%[H]使用
\usepackage{here}

%%%%% cref関連
\usepackage{cleveref}
\crefname{equation}{式}{式}% {環境名}{単数形}{複数形} \crefで引くときの表示
\crefname{figure}{図}{図}% {環境名}{単数形}{複数形} \crefで引くときの表示
\crefname{table}{表}{表}% {環境名}{単数形}{複数形} \crefで引くときの表示
\crefname{algorithm}{Algorithm}{Algorithm}
\newcommand{\crefpairconjunction}{と}
\newcommand{\crefrangeconjunction}{から}
\newcommand{\crefmiddleconjunction}{,}
\newcommand{\creflastconjunction}{,および}
\crefname{chapter}{}{}
\creflabelformat{chapter}{#2#1#3章}
\crefname{section}{}{}
\creflabelformat{section}{#2#1#3節}
\crefname{subsection}{}{}
\creflabelformat{subsection}{#2#1#3小節}
\crefname{subsubsection}{}{}
\creflabelformat{subsubsection}{#2#1#3項}

%%式番号を通し番号に,コメントアウトで節番号付与
\makeatletter
\renewcommand{\theequation}{%
\arabic{equation}}
\makeatother

%%subcaptionの引用時に()をつける
\captionsetup[subfigure]{labelformat=simple}
\renewcommand{\thesubfigure}{(\alph{subfigure})}

%%Timesフォント
\usepackage{times}

%%角度記号パッケージ
\usepackage{mathcomp}

%&table内で\Hlineとすれば太線に
%表作成はこれとかhttps://www.tablesgenerator.com/latex_tables#
\makeatletter
\def\Hline{%
\noalign{\ifnum0=`}\fi\hrule \@height 1pt \futurelet
\reserved@a\@xhline}
\makeatother

%%表セル結合
\usepackage{multirow}

%%単位コマンド簡略化
\newcommand{\m}{\mathrm{m}}
\newcommand{\cm}{\mathrm{cm}}
\newcommand{\km}{\mathrm{km}}
\newcommand{\mm}{\mathrm{mm}}
\newcommand{\s}{\mathrm{s}}
\newcommand{\g}{\mathrm{g}}
\newcommand{\kg}{\mathrm{kg}}
\newcommand{\N}{\mathrm{N}}
\newcommand{\kN}{\mathrm{kN}}
\newcommand{\grad}{\mathrm{grad}}
\newcommand{\rot}{\mathrm{rot}}

%%アブスト体裁調整
\renewcommand{\abstractname}{\if@english Abstract\else 論文要旨\fi}
\makeatletter
\renewenvironment{abstract}{%
\titlepage
% \null     %アブスト上詰め
\vfil
\@beginparpenalty\@lowpenalty
\begin{center}%
    \headfont {\fontsize{13pt}{13pt}\selectfont \abstractname}
    \@endparpenalty\@M
\end{center}}%
{\par\vfil\null\endtitlepage}

%アブスト余白調整
\def\@makechapterhead#1{%
  \vspace*{2\Cvs}% 欧文は50pt,章見出しの上空白
  {\parindent \z@ \raggedright \normalfont
    \ifnum \c@secnumdepth >\m@ne
        \huge\headfont \@chapapp\thechapter\@chappos
        \quad \nobreak % 欧文は20pt
    \fi
    \interlinepenalty\@M
    \huge \headfont #1\nobreak
    \vskip 2\Cvs
    }} % 欧文は40pt
\makeatother

%文字数と行数
\makeatletter
\def\mojiparline#1{
    \newcounter{mpl}
    \setcounter{mpl}{#1}
    \@tempdima=\linewidth
    \advance\@tempdima by-\value{mpl}zw
    \addtocounter{mpl}{-1}
    \divide\@tempdima by \value{mpl}
    \advance\kanjiskip by\@tempdima
    \advance\parindent by\@tempdima
}
\makeatother
\def\linesparpage#1{
    \baselineskip=\textheight
    \divide\baselineskip by #1
}

%目次のインデントを調整
\makeatletter
\renewcommand*{\l@section} {\@dottedtocline{1}{1zw}{2zw}}
\renewcommand*{\l@subsection} {\@dottedtocline{2}{2zw}{4zw}}
\renewcommand*{\l@subsubsection} {\@dottedtocline{3}{3zw}{6zw}}
\makeatother

%svg貼り付け
%Inkscapeをインストール,パスを通す,コンパイル時オプションで"-shell-escape"を"%DOC%"の前につける
\usepackage{svg}
\usepackage{wrapfig}

%本文を複数ファイルに分けてコンパイル&それぞれでコンパイル
\usepackage{subfiles}